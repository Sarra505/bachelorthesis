% !TeX root = ../main.tex
% Add the above to each chapter to make compiling the PDF easier in some editors.
 % write top down, clarify at first what will be written 
 % mention new concept -> connect it to the goal


\chapter{Introduction}\label{chapter:introduction}
In the ever-evolving landscape of computer science and informatics, 
the quest for optimizing query performance in databases remains a 
critical and ongoing challenge. Query optimization lies at the heart 
of efficient database management systems, and its impact reverberates 
throughout a wide range of applications, from e-commerce platforms to 
scientific research. At the core of this challenge is the cardinality 
estimation problem, a fundamental issue that has far-reaching consequences 
in the world of query optimization.

Our bachelor's thesis in informatics embarks on a journey to 
address this challenge by exploring various methods and techniques 
to tackle small linear programming (LP) problems that represent 
the cardinality estimation problem. We endeavor to provide practical 
estimates that guide the selection of data plans for running efficient
 queries, thereby enhancing the overall performance of database systems.

\section{Motivation}

Cardinality estimation, as our research reveals, is a proverbial 
"Achilles' heel" in the domain of query optimization. 
The accuracy of cardinality estimates significantly influences 
the choice of query execution plans, and consequently, 
the efficiency of database queries. Inaccurate estimates 
can lead to suboptimal query plans, causing performance bottlenecks, 
and even system-wide degradation. Therefore, our project aims to 
shed light on this pivotal challenge.

To understand and address the cardinality estimation problem, 
we frame it as a packing linear programming problem. We do so 
with the intent of simulating worst-case scenarios regarding 
join sizes and assessing the query's potential size under such 
conditions. This approach yields hundreds of relatively small 
LP instances that we gather in datasets, each ready to be solved 
using a variety of methods and algorithms.

\section{The Research Approach}

Our research methodology is multifaceted. We systematically 
conduct experiments and benchmarks on these LP problems, drawing 
upon previous work on similar packing LP problems to inform our 
investigations. Through rigorous experimentation, we aim to distill
 insights into the unique characteristics of these LP problems and 
 uncover any underlying patterns in their solution processes.

In the subsequent chapters of this thesis, we delve into the 
background and related work, providing a comprehensive overview
 of linear programming and its various facets. We explore duality, 
 geometric interpretations, feasibility, unboundedness, simplex algorithms,
  runtime complexities, and interior point methods. Additionally, we delve 
  into the AGM bound, which plays a pivotal role in cardinality estimation, 
  and we examine state-of-the-art LP solvers like HIGHS Scipy and Cplex.

\section{Chapter Overview}

This thesis is structured as follows:

\begin{itemize}
  \item \textbf{Chapter 2: Background and Related Work} - Provides a 
  foundational understanding of linear programming, duality, 
  and the intricacies of LP solvers. We also explore cardinality
   estimation and state-of-the-art LP solvers.
  \item \textbf{Chapter 3: Tuning Linear Programming Solvers for Query
   Optimization} - Introduces our proposal and implementation, 
   including the hierarchy, tableau simplex solver, data structures, 
   revised simplex solver, and stability considerations.
  \item \textbf{Chapter 4: Evaluation} - Presents an in-depth analysis 
  of our experiments, including the analysis of the JOB dataset, experiments 
  on randomly generated LPs, and insights into the performance of HIGHS Scipy.
   We also discuss limitations and outline potential avenues for future work.
  \item \textbf{Chapter 5: Conclusion} - Summarizes our findings 
  and contributions, reiterating the importance of cardinality estimation 
  in query optimization, and discusses the implications of our research.
\end{itemize}

Throughout this thesis, we aim to provide a comprehensive 
examination of the cardinality estimation problem and the methods 
to solve it using linear programming. Our ultimate goal is to contribute 
valuable insights that can enhance the time and memory complexity of the 
query optimization process.
