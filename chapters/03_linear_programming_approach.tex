% !TeX root = ../main.tex
% Add the above to each chapter to make compiling the PDF easier in some editors.

\chapter{Tuning Linear Programming Solvers for Query Optimization}\label{chapter:linearprogramming}
This is the body
\section{Proposal}

\section{Experimental Design}
\subsection{Analysis of dataset properties}
In  this subsection we will conduct an analysis of our dataset properties. What are the
particularites of the structure of these LP problems, is their any patterns in their solution
process. This anaylsis is based on observing the statistical results we obtained from
running different solvers on these problems. This will later provide us with insight
regarding optimization of these problems.
TPC-H is a Decision Support Benchmark
The TPC-H is a decision support benchmark. It consists
of a suite of business oriented ad-hoc queries and concurrent data modifications.
The performance metric reported by TPC-H is called the TPC-H Composite
Query-per-Hour Performance Metric (QphH@Size)

\begin{algorithm}
    \caption{Tableau Simplex Algorithm}
    \begin{algorithmic}[1]
        \State \textbf{Input:} Packing LP maximisation problem in computational form
        \State \textbf{Output:} Optimal value $z$

        \State \textbf{Step 1:} Pricing: Find pivot column, or entering variable using Bland's rule
        \State \hspace{\algorithmicindent} $enteringVars \gets \text{findPivotColumnCandidates}()$
        \State \hspace{\algorithmicindent} \textbf{if} no entering variable found \textbf{then}
        \State \hspace{\algorithmicindent} \hspace{\algorithmicindent} \text{print} "Optimal value reached."
        \State \hspace{\algorithmicindent} \hspace{\algorithmicindent} \Return $z$
        \State \hspace{\algorithmicindent} \textbf{end if}
        \State \hspace{\algorithmicindent} $pivotColumn \gets enteringVars[0]$

        \State \textbf{Step 2:} Find pivot row, or leaving variable using the ratio test
        \State \hspace{\algorithmicindent} $pivotRow \gets \text{findPivotRow}(pivotColumn)$
        \State \hspace{\algorithmicindent} \textbf{if} no leaving variable \textbf{then}
        \State \hspace{\algorithmicindent} \hspace{\algorithmicindent} \text{print} "The given LP is unbounded."
        \State \hspace{\algorithmicindent} \hspace{\algorithmicindent} \Return $\infty$
        \State \hspace{\algorithmicindent} \textbf{end if}

        \State \textbf{Step 3:} Update the tableau using pivotting and update the objective function value
        \State \hspace{\algorithmicindent} $\text{doPivotting}(pivotRow, pivotColumn, z)$

        \State \textbf{Goto Step 1}
    \end{algorithmic}
\end{algorithm}

\subsection{Dataset Structure}
Our dataset stucture:
as opposed to what the linear programming research has dealt with, which is
very large problems, we are dealing with hundreds of small problems. These are represented
in the revised simplex algorithm by
sparse matrices but not as sparse as it would have been if the problem was large,
small matrices that are not small enough to be dense.
(they still have quite a number of non-zeroes).

\section{Analysis}
Some metrics:
\begin{itemize}
    \item number of iterations
    \item runtime
    \item number of loops ?
    \item for matrix : number of columns and rows, nonzeros and density
\end{itemize}

\section{Results}
All the following results have been obtained on a personal computer with AMD 4000 series
RYZEN, 16GB RAM running Ubuntu. Using the following settings:
\begin{itemize}
    \item Presolve techniques are not used
    \item scaling techniques are not used
    \item The computed optimal solutions have been validated using the scipy python library.
\end{itemize}