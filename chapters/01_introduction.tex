% !TeX root = ../main.tex
% Add the above to each chapter to make compiling the PDF easier in some editors.
 % write top down, clarify at first what will be written 
 % mention new concept -> connect it to the goal
 % python scify different algos (interior piint vs. simplex)
 % metrics : numbeer of steps, loops...
 % what s special about dataset

\chapter{Introduction}\label{chapter:introduction}

Our aim with this project is to investigate and compare different methods 
and techniques to solve small linear programming problems representing 
the problem of cardinality estimation. Our goal is to 
estimate realistic and useful upper bounds on query sizes. Studies have 
shown that cardinality estimation is the major root of many issues in 
query optimization\parencite{ngo2022information}. And yet, theoretical upper bounds that
are way too large would be useless since we want practical estimation to choose
the best from data plans to run efficient queries.
For this purpouse, we will introduce a formal description of the cardinality
estimation problem, represent it in the form of a packing linear programming problem with
the intention of maximizing the size of the query under some constraints. The result is
hundreds of relatively small LP that we collect in datasets and solve them 
with different methods and algorithms. We then draw conclusions based on the results of
our experiments, benchmarks and the previous work done on similar packing LP problems.
This should guide us into constructing a thorough analysis of the particularities of
these LP problems, what's unique about their structure and if their solution
process is following any patterns. We then discuss and draw hypotheses on the ways this 
analysis can be exploited to further optimize the solution process: which methods or 
combination of methods deliver the best time and memory complexity.
