% !TeX root = ../main.tex
% Add the above to each chapter to make compiling the PDF easier in some editors.

\chapter{Related work}\label{chapter:relatedwork}

\section{Background}
or Fundamentals: the knowledge the reader needs to understand my contribution, mostly definition of mathematical concepts needed
\subsection{Linear Programming}

\subsection{Cardinality Estimation}
Defining the problem of upper bounding a multi-join query size as a packing linear programming problem.
To illustrate the main ideas, we start with an example where the query is a simple join
between two relations
\[
Q(X, Y, Z) = R(X, Y) \land S(Y, Z)
\]
In the context of our packing LP problem, we start with the inequality \ref{eq:initial_inequality}. Applying the natural logarithm to both sides yields \ref{eq:log_inequality}. We then rename the variables, simplifying the inequality to \ref{eq:renamed_inequality}. 
Normalizing by dividing both sides by \(r'\), we obtain \ref{eq:normalized_inequality}. This leads us to the objective function for our packing LP problem.
\begin{align}
    |a| \cdot |b| &\leq |R| \label{eq:initial_inequality} \\
    \ln|a| + \ln|b| &\leq \ln|R| \label{eq:log_inequality} \\
    a' + b' &\leq r' \label{eq:renamed_inequality} \\
    \frac{1}{r'} a' + \frac{1}{r'} b' &\leq 1 \label{eq:normalized_inequality} \\
    \text{maximize } a' + b' + c' + d' \quad &\text{s.t.} \quad \frac{1}{r'} a' + \frac{1}{r'} b' \leq 1 \label{eq:objective_function}
    \end{align}
\subsection{The Simplex Algorithm}

\section{Previous Work}
alternative approaches that are superseded by my work
\subsection{Comparative studies of different update methods}
\subsection{Other techniques}
