% !TeX root = ../main.tex
% Add the above to each chapter to make compiling the PDF easier for some editors.
 % Write top-down, clarify at first what will be written 
 % mention new concept -> connect it to the goal


 \chapter{Introduction}\label{chapter:introduction}
 In this chapter we motivate the purpose of our research and present an overview of its structure, highlighting the importance of reliable cardinality estimation in the field of query optimization. Our contribution's relevance lies in selecting the best method or combination of methods to solve hundreds of small LP problems efficiently, optimizing execution time and memory usage. The cardinality estimation problem can be solved using linear programming, so our research opens doors for a faster cardinality estimation, and, in turn, efficient query execution.
 
 \section{Motivation}
 
 The quest for optimizing query execution in databases is a critical
 challenge. The query optimizer attempts 
 to determine the most efficient way to execute a given query by considering the possible 
 query plans \parencite{leis2018query}. 
 These plans are all equivalent in their final output but vary in their
 cost, that is, the amount of time that they need to run.
 
 Cost-based query optimizers select query plans that have the lowest estimated cost to
 execute. One parameter that plays an important role in this selection process 
 is the cardinality estimates 
 of these plans. Therefore, improving cardinality estimation leads to better estimated costs 
 and, in turn, faster execution plans \parencite{graefe1993volcano}.
 
 Linear Programming (LP) is a mathematical modeling technique to calculate the optimum of a linear function
 given a set of linear constraints \parencite{chvatal1983linear}.
 Cardinality estimation can be represented through linear programming \parencite{atserias2013size}.
 
 Our aim with this project is to implement, 
 investigate, and compare different methods suitable for solving hundreds of small LP 
 problems. These methods include the standard simplex algorithm and the revised simplex algorithm.

 We test our methods alongside Umbra's LP solver for cardinality estimation \parencite{neumann2020umbra}. 
 By running various benchmarks, we conduct an analysis of the structure of these LP problems, compare the time profiles for the different methods tested and finally 
 make a recommendation on the methods or combination of methods that deliver the best time and memory complexity.
 
 \section{Chapter Overview}
 
 This thesis is structured as follows:
 
 \begin{itemize}
   \item \textbf{Chapter 2: Background and Related Work} - Provides a 
   foundational understanding of linear programming, duality, 
   and an overview of the different LP solvers. We also explore cardinality
    estimation and state-of-the-art LP solvers.
   \item \textbf{Chapter 3: Tuning Linear Programming Solvers for Query
    Optimization} - Introduces our proposal and implementation, 
    including the hierarchy, tableau simplex solver, data structures, 
    revised simplex solver, and stability considerations.
   \item \textbf{Chapter 4: Evaluation} - Presents our results as well as
   an in-depth analysis 
   of our results, including the analysis of the query dataset, and the
    randomly generated LP problems.
   \item \textbf{Chapter 5: Conclusion} - Summarizes our findings 
   and contributions.  We also discuss the implications of our research. and outline potential future work.
   and discusses 
 \end{itemize}
 
 Throughout this thesis, we aim to provide a comprehensive 
 examination of the cardinality estimation problem and the methods 
 to solve it using linear programming. Our ultimate goal is to contribute 
 valuable insights that can enhance the
 query optimization process.
 