\chapter{\abstractname}

Our aim with this project is to implement, 
investigate and compare different methods 
and techniques to solve small linear programming problems representing
 the problem of cardinality estimation. A way to estimate realistic and
useful upper bounds of query sizes is through linear programming.
 Studies have 
shown that cardinality estimation is the major root of many issues in 
query optimization, which is why we want a 
practical estimate to choose
the best from data plans to run efficient queries.
For this purpouse, the cardinality
estimation problem can be represented in the form of a packing linear programming problem with
the intention of assuming worst-case join sizes and seeing how
large the query can get. The result is
hundreds of relatively small LP that we collect in datasets and solve them 
with different methods and algorithms. We then draw conclusions based on the results of
our experiments, benchmarks and the previous work done on similar packing LP problems.
This should guide us into constructing a thorough analysis of the particularities of
these LP problems, what's unique about their structure and if their solution
process follows any patterns. We then discuss and draw hypotheses on the ways this 
analysis can be exploited to further optimize the solution process: which methods or 
combination of methods deliver the best time and memory complexity.



